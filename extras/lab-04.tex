\documentclass{tufte-handout}

\usepackage{xcolor}

% set image attributes:
\usepackage{graphicx}
\graphicspath{ {images/} }

% set hyperlink attributes
\hypersetup{colorlinks}

% set table attributes
\usepackage{tabu}
\usepackage{booktabs}

% set list attributes:
\usepackage{enumerate}
\usepackage{enumitem}


% ========================================================

% define the title
\title{SOC 4650/5650: Lab-04 - Mapping Structures at Risk of Severe Weather Using R}
\author{Christopher Prener, Ph.D.}
\date{Fall 2018}
% =======================================================
\begin{document}
% =======================================================
\maketitle % generates the title
% =======================================================

\section{Directions}
Using data from the \texttt{data/lab-04/} folder available in the \texttt{lecture-05} repository, create several maps using RStudio. Your entire project folder system should be uploaded to GitHub by Monday, February 18\textsuperscript{th} at 4:15pm.

\vspace{5mm}
\section{Analysis Development}
The goal of this section is to create a self contained project directory with all of the data, code, map documents, results, and documentation a project needs.

\vspace{3mm}
\subsection{Download Data}
\begin{enumerate}[label=\alph*.]
\item \textbf{Clone} the \texttt{lecture-05} repository from GitHub using GitHub Desktop.\sidenote{If you are not sure where your GitHub Desktop data has download to on the computer, you can \textsf{right click} on the repo's entry in GitHub Desktop and have it take you to the repo in Windows File Explorer. By default, this should be within your \texttt{Documents/} directory.}
\end{enumerate}

\vspace{3mm}
\subsection{Create a Project Folder System}
\begin{enumerate}[label=\alph*.]
\setcounter{enumi}{1}
\item Using RStudio, add an R Project to the \textit{existing} directory in your assignments repository named \texttt{Lab-04}. To do this, you will want to go to: \textsf{File $\triangleright$} {\color{red}\textsf{New Project}} \textsf{$\triangleright$ Existing Directory} and find your \textit{existing} \texttt{Lab-04} folder.
\item In the \textsf{Files} tab on the lower right-hand side of RStudio's screen, add a New Folder using the \textsf{New Folder} button right below \textsf{Files}. Name this new folder \texttt{docs}. Add two others named \texttt{data} and \texttt{results}.
\item Reduce RStudio for a moment. Using the Windows File Explorer app, find your project as well as the repository you cloned previously. It is easiest if these are in two separate windows.
\item Drag the lab data from \texttt{lecture-05/data/lab-04/} into your RStudio Project's \texttt{data/} subdirectory. Verify using RStudio that all of these data are accessible from within your project.
\end{enumerate}

\vspace{3mm}
\subsection{Create an R Markdown File}
\begin{enumerate}[label=\alph*.]
\setcounter{enumi}{5}
\item Back in RStudio, create a new notebook by going to \textsf{File $\triangleright$ New File $\triangleright$} {\color{red}\textsf{R Markdown}}. Choose the SLU Sociology template and save it within that \texttt{docs/} subdirectory you just created. The notebook should be named \texttt{lab-04}.
\item Expand the YAML heading by adding your name and the assignment title ``Lab 04''.
\item Use RMarkdown syntax to create your first assignment notebook! Make sure it has an introductory section, a section for loading packages, a section for loading data, and a section for part 2 below. These sections should be second-level headings (e.g. \texttt{\#\# Introduction}). Within Parts 1 through 4, use third level headings to designate question numbers (e.g. \texttt{\#\#\# Question 1}).
\item When you are done, ``knit'' your document by clicking the \textsf{Knit} button in the toolbar at the top of the notebook.
\end{enumerate}

\vspace{3mm}
\subsection{Load Data}
\begin{enumerate}[label=\alph*.]
\setcounter{enumi}{9}
\item Import the files \texttt{METRO\_STRUCTURE\_PctMobileHome.shp} and \\ \texttt{METRO\_BOUNDARY\_Counties.shp} into your global environment.
\end{enumerate}

\vspace{5mm}
\section{Part 1: Data Exploration}
The goal of this section is to create an interactive map using \texttt{leaflet}. 
\begin{enumerate}
\item List the variables in the mobile home data using \texttt{str()}.
\item Produce an interactive visual preview of the data using the \texttt{mapview} package.
\end{enumerate}

\vspace{5mm}
\section{Part 2: Interactive Mapping}
The goal of this section is to create an interactive map using \texttt{leaflet}. 
\begin{enumerate}
\setcounter{enumi}{2}
\item Use \texttt{names(providers)} to identify a basemap to use for your map.
\item Using a color ramp and basemap of your choice, symbolize the percentage of mobile homes in each county. Add a popup that has the county name as well as the percentage of mobile homes in that county on separate lines. The popup should be formatted like the ones we created during the lecture, with bold labels on separate lines for each characteristic on the pop-up. Your map should also have a legend and the legend's name should be edited for clarity. \sidenote{\textit{Hint:} There is no need to export this - make sure you have PhantomJS installed on your computer and it will be included in your final, knit document.}
\end{enumerate}

\vspace{5mm}
\section{Part 3: Static Mapping for Digital Use}
The goal of this section is to create a static map using \texttt{ggplot2}. 
\begin{enumerate}
\setcounter{enumi}{4}
\item Using a color ramp of your choice (though it should be different from your interactive map), symbolize the percentage of mobile homes in each county. Use the metro counties boundary layer overlaid on your choropleth layer to more clearly mark the county boundaries. Make sure you edit your legend's title and add a title, subtitle, and a caption to the plot. Your caption should include your name.
\item Export your map as a \texttt{.png} file (suitable for use online or in another digital medium) at \texttt{500} dots per inch. Make sure your map is saved to the \texttt{results} subfolder of your project.
\end{enumerate}

\vspace{5mm}
\section{Part 4: Static Mapping for Print Use}
The goal of this section is to create a static map using \texttt{tmap}. 
\begin{enumerate}
\setcounter{enumi}{6}
\item Using a color ramp of your choice (though it should be different than your prior two maps), symbolize the percentage of mobile homes in each county. Use the metro counties boundary layer overlaid on your choropleth layer to more clearly mark the county boundaries. Make sure you edit your legend's title and add a title, and scalebar to the plot.
\item Export your map as a \texttt{.pdf} file (suitable for print) at \texttt{500} dots per inch. Make sure your map is saved to the \texttt{results} subfolder of your project.
\item Adapt the code from Question 7 and add a histogram to your legend. Again, use a different color ramp for your data than you have used previously. If you have yet to use \texttt{RColorBrewer} or \texttt{viridis}, use a palette from the unused package here.
\item Export your map as a \texttt{.pdf} file (suitable for print) at \texttt{500} dots per inch. Make sure your map is saved to the \texttt{results} subfolder of your project.
\end{enumerate}

\vspace{5mm}
\section{Analysis Development Follow-up}
Don't forget to knit your document when you are done!


% =======================================================
\end{document}